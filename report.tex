\documentclass[12pt]{article}

\usepackage{geometry}
 \geometry{
 a4paper,
 total={170mm,257mm},
 left=20mm,
 top=20mm,
 }


\begin{document}


\title{EFFECTS OF PHYSICAL EXERCISES}
\author{MATOVU JOSEPH  215011917 ;15/U/7462/PS} 
\date{29th march 2017}
\maketitle

\begin{abstract}
This report investigates the effects of physical exercises to human life. This includes the health and physical effects.
This report also looks at the different types of exercises, their importance and the activities involved in each type.
\end{abstract}

\tableofcontents

\section{INTRODUCTION}
Physical exercise is any bodily activity that enhances or maintains the physical fitness and overall health and wellness. Physical exercises can be done during free times. This can be during early mornings, evening, sometimes once a week depending on someone’s schedule. 

\section{MAJOR TYPES OF EXERCISES}

\subsection{Aerobics}
This type benefits the heart and lungs most and in this case, it makes the heart to beat harder and breathe harder. Under this type of exercise, we have: - running, swimming, walking, bicycling, dancing, jumping, and so on.

\subsection{Muscle strengthening}
This type improves the strength, power, and endurance of muscles. This type of exercise can also involve some aerobics like climbing stairs. Other activities include: doing pushups, sit-ups, lifting weights, and digging as well.

\subsection{Bone strengthening}
In this type, your feet, legs or arms support your body weight and the muscles push against the bones. This helps the bones to get strong. Some activities include: running, weight lifting and jumping rope.

\subsection{Stretching}
This type helps to improve one’s flexibility and ability to fully move the joints. Activities under this type include: - touching the toes, doing side stretches, and doing yoga exercises.

\section{EFFECTS OF PHYSICAL EXERCISES}

\subsection{Health effects}

\begin{itemize}
\item 	Prevention of heart disease and stroke by strengthening the heart muscleand lowering blood pressure which also increases the heart working capacity.

\item	Reduction of body fat which could have resulted to high blood pressure.

\item 	 Physicals also help in controlling diabetes.

\item 	 By increasing muscle strength and endurance and improving flexibility, this helps to prevent back pain.

\item	 Physical exercises also help in controlling weight and prevent obesity a major risk factor of many diseases.

\end{itemize}

\subsection{Physical effects}
\begin{itemize}
\item  	 Physical exercises help to maintain the body shape and size.

\item	 Increases one’s strength and power.

\item	 Enhancement of one’s muscles.

\item	 Brings about flexibility and ability to move joints.

\end{itemize}

\section{Conclusion}
This report has identified the major types of physical exercises which bring about the health and physical benefits to human life. There for a health life, people should try fixing times in their daily schedules to do physical exercises for a health wellbeing of their bodies and physical maintenance of their bodies as well. 


\end{document}